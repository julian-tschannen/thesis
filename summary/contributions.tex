%############################################################################
\subsection{Overview of Contributions}
%############################################################################

This thesis makes the following contributions:

%----------------------------------------------------------------------------
\subsubsection{Design of an integrated verification environment~\cite{TSCHANNEN11}}
%----------------------------------------------------------------------------

We have developed an approach to integrate verification tools in an IDE, resulting in an \emph{integrated verification environment}.
The high-level approach produces correctness scores for each routine and class, based on a combination of correctness scores of individual tools, confidence in the tool's results, and other factors such as importance of routines.
We have implemented this method in the \VAssist as part of the Eiffel Verification Environment.


%----------------------------------------------------------------------------
\subsubsection{Auto-active verification of Eiffel~\cite{TSCHANNEN15,TSCHANNEN14}}
%----------------------------------------------------------------------------

We have developed \AutoProof, an auto-active verifier for Eiffel.
The verifier covers a large part of the Eiffel language and can be used to verify challenging problems by supporting semantic collaboration~\cite{POLIKARPOVA14} and verification of function objects~\cite{NORDIO10}.
\AutoProof has been used to verify a general-purpose container library~\cite{POLIKARPOVA15}, verify client code of said library, as well as for teaching in a verification course~\cite{FURIA15}.
We discuss in-depth how \AutoProof verifies challenging examples and provide an online repository of examples verified with \AutoProof.
\AutoProof is integrated in the Eiffel Verification Environment and available through a web interface.


%----------------------------------------------------------------------------
\subsubsection{Methodologies for auto-active verification~\cite{TSCHANNEN13,TSCHANNEN12}}
%----------------------------------------------------------------------------

We introduce, implement, and evaluate on realistic examples several general methodologies usable by auto-active verifiers.

\begin{itemize}
\item
\emph{Two-step verification}, a technique to improve the feedback of failed verifications by doing a second verification step that uses inlining and unrolling.

\item
A methodology to use contracts of the dynamic type whenever the verifier can infer the dynamic type at a program location.

\item
A methodology to annotate and verify Eiffel code that uses exceptions.

\end{itemize}




