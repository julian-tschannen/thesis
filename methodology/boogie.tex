%############################################################################
\section{A Short Introduction to Boogie} \label{boogie-intro}
%############################################################################


Boogie is a language for verification~\cite{LEINO08}, as well as an
automated verifier that takes programs written in the Boogie language
as input.  \AutoProof verifies Eiffel programs by translating them
into the Boogie language and then by calling the Boogie verifier on
the translation. The Boogie verifier generates verification conditions
for the input program, and supports different prover back-ends (e.g.,
Z3 and Simplify) to discharge them.  For readers unfamiliar with
Boogie, this section describes the essential features of the Boogie
language used in the rest of the paper.

The Boogie language offers two kinds of constructs: a simple
imperative modular programming language---used to translate the source
program (Eiffel, in our case)---and a specification language based on
first-order logic---used to define specification elements
and background logic theories needed to support complex
specifications.

Boogie's specification language is a typed first-order logic with arithmetic.
The basic types include Booleans (\b{bool}) and mathematical (unbounded) integers (\b{int}); the type constructors support the definition of derived types.
Line~\ref{pl1} in Figure~\ref{fig-boogie-intro-1} declares a new type \b{person}, which Boogie treats as a fresh sort for variables.
The specification language supports the definition of global variables and constants, functions (in the sense of mathematical logic), and axioms.
Line~\ref{pl2}, for example, declares a global constant \b{eve} of type \b{person}.
Lines~\ref{pl3} and \ref{pl4} declare two functions \b{age} and \b{can_vote}.
Lines~\ref{pl5} and \ref{pl6} introduce two axioms about the declared items: \b{age} is defined as \b{23} for argument \b{eve}; and \b{can_vote} is true precisely for persons whose age is greater than or equal to \b{18}.


\begin{bfigure}[ht]{Some definitions in Boogie's specification language.}{fig-boogie-intro-1}
#\label{pl1}#type person;
#\label{pl2}#const eve: person;
#\label{pl3}#function age(p: person) returns (int);
#\label{pl4}#function can_vote(p: person) returns (bool);
#\label{pl5}#axiom (age(eve) = 23);
#\label{pl6}#axiom (forall p: person :: can_vote(p) <==> age(p) >= 18);
\end{bfigure}

Boogie's programming language supports the definition of procedures.
Each procedure has a signature, which may include a specification in
terms of preconditions (\b{requires}), postconditions
(\b{ensures}), and frame clauses (\b{modifies}).  The
specification clauses contain formulas in Boogie's specification
language.  Postconditions, in addition, supports the usage of the
\b{old} keyword to evaluate expressions in the state
\emph{before} a procedure was called.  Modifies clauses, instead,
define a procedure's \emph{frame}, that is the set of global variables
the procedure may modify.  Pre- and postconditions may be marked as
\b{free}, which prevents the generation of proof obligations
based on them: a \b{free} assertion is assumed to hold whenever
convenient, but need not be checked when required.

Procedure implementations use standard imperative constructs
(assignments, conditionals, loops, jumps, and procedure calls) with
the usual semantics.  To write nondeterministic programs, Boogie's
programming language includes the \b{havoc} command, which
assigns a nondeterministically chosen value to its argument variables.
To constrain the effects of \b{havoc} and to express intermediate
verification conditions, Boogie's programming language also offers
\b{assert} and \b{assume} statements.  Both take an
arbitrary formula \b{F} as argument. The program state of every
execution reaching an \b{assert F} must satisfy \b{F};
otherwise, verification fails.  Conversely, the verification process
can assume that \b{F} holds of the program state whenever an
execution reaches \b{assume F}, which ``shapes'' the
nondeterministic behavior when convenient.


Figure~\ref{fig-boogie-intro-2} shows the specification and implementation of a procedure \b{vote} to cast a vote, demonstrating Boogie syntax.

\begin{bfigure}[ht]{A Boogie procedure \b{vote}: specification and implementation.}{fig-boogie-intro-2}
var votes: int;

procedure vote(p: person);
	requires can_vote(p);
	ensures votes = old(votes) + 1;
	modifies votes;

implementation vote(p: person) {
	votes := votes + 1
}
\end{bfigure}
