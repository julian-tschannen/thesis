%############################################################################
\section{Polymorphic Calls}
\label{sec:m-polymorphic}
%############################################################################


The redefinition of a routine $r$ in a descendant class can \emph{strengthen} $r$'s original postcondition by adding an \e{ensure then} clause, which conjoins the postcondition in the precursor.
The example in Figure~\ref{example-exp} illustrates a difficulty occurring when reasoning about postcondition strengthening in the presence of polymorphic types.
The deferred (abstract) class \e{EXP} models nonnegative integer expressions and provides a routine \e{eval} to evaluate the value of an expression object; even if \e{eval} does not have an implementation in \e{EXP}, its postcondition specifies that the evaluation always yields a nonnegative value stored in attribute \e{last_value}, which is set as side effect.
Classes \e{CONST} and \e{PLUS} respectively specialize \e{EXP} to represent integer (nonnegative) constants and addition.
Class \e{ROOT} is a client of the other classes, and its \e{main} routine attaches an object of subclass \e{CONST} to a reference with static type \e{EXP}, thus exploiting polymorphism.
Similar issues occur when a descendant class \emph{weakens} a some routine $r$'s precondition with an \e{require else} clause.


\begin{figure}[!ht]
\begin{tabular}{ll}

{\begin{erunning}
deferred class EXP
feature
	last_value: INTEGER
	eval
		deferred
		ensure
			last_value >= 0
		end
end
 #\ # 
 #\ # 
 #\ # 
 #\ # 
\end{erunning}}
&
\hspace{5mm}
{\begin{erunning}
class PLUS inherit EXP feature 
	left, right: EXP
	eval do
			left.eval; right.eval
			last_value := left.last_value + 
						  right.last_value
		ensure then
			last_value = left.last_value + 
						 right.last_value
		end
invariant
	left /= right /= Current
end
\end{erunning}}
\vspace{5mm}
\\
{\begin{erunning}
class CONST inherit EXP
feature 
	value: INTEGER
	eval do
			last_value := value
		ensure then
			last_value = value
		end
invariant
	value >= 0
end
\end{erunning}}
&
\hspace{5mm}
{\begin{erunning}
class ROOT
feature    
	main
		local
				e: EXP
		do
			e := create {CONST}.make (5);
			e.eval
			check e.last_value = 5 end    
		end
end 
\end{erunning}}

\\

\end{tabular}
\caption{Nonnegative integer expressions.}\label{example-exp}
\end{figure}

The verification goal is proving that, after the invocation $e.eval$ (in class \e{ROOT}), \e{eval}'s postcondition in class \e{CONST} holds, which subsumes the \e{check} statement in the caller.
Reasoning about the invocation only based on the static type \e{EXP} of the target \e{e} guarantees the postcondition \e{last_value >= 0}, which is however too weak to establish that \e{last_value} is exactly 5.

Other approaches, such as M\"uller's~\cite{MUELLER02}, have targeted these issues in the context of Hoare logics, but they usually are unsupported by automatic program verifiers.
In particular, with the Boogie translation of polymorphic assignment implemented in Spec\#, the assertion \e{check e.last_value = 5 end} in class \e{ROOT} can be verified only if \e{eval} is declared \emph{pure}; \e{eval} is, however, not pure.
The Spec\# methodology selects the pre and postconditions according to static types for non-pure routines: the call \e{e.eval} only establishes \e{e.last_value >= 0}, not the stronger \e{e.last_value = 5} that follows from \e{e}'s dynamic type \e{CONST}, unless an explicit cast redefines the type \e{CONST}.
The rest of the section describes the solution implemented in \AutoProof, which handles contracts of redefined routines.




%============================================================================
\subsection{Polymorphism in Boogie}
%============================================================================

The Boogie translation implemented in \AutoProof can handle polymorphism appropriately even for non-pure routines; it is based
on a methodology for agents~\cite{NORDIO10} and on a methodology for pure routines~\cite{DARVAS07,LEINO08b}.
The rest of the section discusses how to translate postconditions and preconditions of redefined routines in a way that accommodates polymorphism, while still supporting modular reasoning.


%----------------------------------------------------------------------------
\subsubsection{Postconditions}
%----------------------------------------------------------------------------

The translation of the postcondition of a routine $r$ of class $X$ with result type $T$ (if any) relies on an auxiliary function \b{post.X.r}:
\begin{brunning}
function post.X.r (h1, h2: HeapType; c: ref; res: T) returns (bool);
\end{brunning}
which predicates over two heaps (the pre and post-states in $r$'s postcondition), a reference \b{c} to the current object, and the result \b{res}.
$r$'s postcondition in Boogie references the function \b{post.X.r}, and includes the translation $\nabla_{post}(X.r)$ of $r$'s postcondition clause syntactically declared in class \e{X}:\footnote{The translation differs for calls to \e{Precursor} (\lstinline[language=Java]|super| in Java and \lstinline[language={[Sharp]C}]|base| in C\#).}
\begin{brunning}
procedure X.r (Current: ref) returns (Result: T);
  free ensures post.X.r (Heap, old(Heap), Current, Result);
  ensures $\nabla_{post}(X.r)$;
\end{brunning}
\b{post.X.r} is a \b{free ensures}, hence it is ignored when proving $r$'s implementation and is only necessary to reason about usages of $r$.

The function \b{post.X.r} holds only for the type $X$; for each class $Y$ which is a descendant of $X$ (and for $X$ itself), an axiom links $r$'s postcondition in $X$ to $r$'s strengthened postcondition in $Y$:
\begin{brunning}
axiom (forall h1, h2: HeapType; c: ref; r: T ::
    type_of(c) <: Y ==> (post.X.r(h1, h2, c, r) ==> #$\nabla_{post}(Y.r)$#));
\end{brunning}
The function \b{type_of} returns the type of a given reference; hence the postcondition predicate $post.X.r$ implies an actual postcondition $\nabla_{post}(Y.r)$ according to $c$'s dynamic type. 

In addition, for each redefinition of $r$ in a descendant class $Z$, 
the translation defines a fresh Boogie procedure $Z.r$ with 
corresponding postcondition predicate $post.Z.r$ and axioms 
for all of $Z$'s descendants.


%----------------------------------------------------------------------------
\subsubsection{Preconditions}
%----------------------------------------------------------------------------


Eiffel also supports \emph{weakening of preconditions}. Therefore, the precondition
of a routine can also depend on the dynamic type. We use a similar
translation as for the postcondition. Given a routine $r$ of type $X$, 
a precondition predicate is generated and used in the signature 
of the generated Boogie procedure:
\begin{brunning}
function pre.X.r(h: HeapType; c: ref) returns (bool);
\end{brunning}
which predicates over one heap and a reference \e{c} to the current object.
\e{r}'s precondition in Boogie references the function \b{pre.X.r}, and it includes the translation $\nabla_{pre}(X.r)$ of $r$'s precondition originally declared in class $X$:
\begin{brunning}
procedure X.r (Current: ref) returns (Result: T);
	requires pre.X.r(Heap, Current);
	free requires #$\nabla_{pre}(X.r)$#
\end{brunning}
Conversely to the postcondition, establishing $r$'s precondition is a responsibility of callers of $r$; clients have to establish the precondition determined by the dynamic type---captured by the function \b{pre.X.r}---, whereas the precondition originally given in $X$ is given as a \b{free requires} and is only used to prove $r$'s implementation.
\begin{brunning}
axiom (forall h: HeapType; c: ref ::
		type_of(c) <: Y ==> (#$\nabla_{pre}(Y.r)$# ==> pre.X.r(h, c)));
\end{brunning}
To establish \b{pre.X.r}, it is enough to establish any of the clauses $\nabla_{pre}(Y.r)$.


\begin{bfigure}[p]{Simplified Boogie translation of the Eiffel classes in Figure~\ref{example-exp}.}{fig-exp-boogie}
function post.EXP.eval(h1, h2: HeapType; c: ref) returns (bool);

procedure EXP.eval(current: ref); #\label{l:exp:boogie:a}#
	free ensures post.EXP.eval(Heap, old(Heap), current); #\label{l:exp:boogie:k}#
	ensures Heap[current, last_value] >= 0;
	// other specification omitted #\label{l:exp:boogie:b}#

axiom (forall h1, h2: HeapType; o: ref :: #\label{l:exp:boogie:c}#
	type_of(o) <: EXP ==> 
	(post.EXP.eval(h1, h2, o) ==> (h1[o, last_value] >= 0)) ); #\label{l:exp:boogie:d}#
axiom (forall h1, h2: HeapType; o: ref :: #\label{l:exp:boogie:e}#
	type_of(o) <: CONST ==> 
	(post.EXP.eval(h1, h2, o) ==> h1[o, last_value] == h1[o, value]) ); #\label{l:exp:boogie:f}#
axiom (forall h1, h2: HeapType; o: ref :: #\label{l:exp:boogie:g}#
	type_of(o) <: PLUS ==> 
	(post.EXP.eval(h1, h2, o) ==> (h1[o, last_value] == 
		h1[h1[o, left], last_value] + h1[h1[o, right], last_value])) ); #\label{l:exp:boogie:h}#

implementation ROOT.main (Current: ref) {
		var e: ref;
	entry:
		// translation of: #\textbf{create} \{\textit{CONST}\} \textit{e.make} (5)#
		havoc e;
		assume Heap[e, allocated] == false;
		Heap[e, allocated] := true;
		assume type_of(e) == CONST; #\label{l:exp:boogie:i}#
		call CONST.make(e, 5);
		// translation of: #\textit{e.eval}#
		call EXP.eval(e);
		// translation of: #\textbf{check} \textit{e.last\_value} = 5 \textbf{end}#
		assert Heap[e, last_value] == 5; #\label{l:exp:boogie:j}#
		return;
}
\end{bfigure} 



%============================================================================
\subsection{An Example of Polymorphism with Postconditions}
%============================================================================

Figure~\ref{fig-exp-boogie} shows the essential parts of the Boogie translation of the example in Figure~\ref{example-exp}. 
The translation of routine \e{eval} in lines \ref{l:exp:boogie:a}--\ref{l:exp:boogie:b} references the function \b{post.EXP.eval}; the axioms in lines \ref{l:exp:boogie:c}--\ref{l:exp:boogie:h} link that function to $r$'s postcondition in \e{EXP} (lines \ref{l:exp:boogie:c}--\ref{l:exp:boogie:d}) and to the additional postcondition introduced in \e{CONST} (lines \ref{l:exp:boogie:e}--\ref{l:exp:boogie:f}) and \e{PLUS} (lines \ref{l:exp:boogie:g}--\ref{l:exp:boogie:h}) for the same routine.

The rest of the figure shows the translation of the client class \e{ROOT}. To verify the assertion on line~\ref{l:exp:boogie:j} the verifier will use the fact that the reference \b{e} is of type \e{CONST} (line~\ref{l:exp:boogie:i}), that the postcondition function \b{post.EXP.eval} holds thanks to the postcondition of \e{eval} (line~\ref{l:exp:boogie:k}), and the axiom generated for the specific subtype (line~\ref{l:exp:boogie:e}). With this information, the verifier can establish that the assertion holds.


%============================================================================
\subsection{Related Work}
%============================================================================


Spec\#~\cite{BARNETT05} uses dynamic postconditions for pure functions, taking the dynamic object type into account. In contrast, our approach works for functions that have side-effects and for adaptations of preconditions as well.

Rapid Type Analysis~\cite{BACON96} can be used to determine all possible types of an object at a call site.
The algorithm analyses the full system and is therefore non-modular.
For local variables that are only available in the context of a single routine this analysis would be suitable even for modular verification.

Using abstract interpretation~\cite{COUSOT77}, Rapid Atomic Type Analysis~\cite{LOGOZZO10} can be used to infer precise types of numeric variables, even for dynamically typed languages.

