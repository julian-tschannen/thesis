%############################################################################
\section{Input Language Support}
\label{sec:ap-coverage}
%############################################################################


\AutoProof supports most of the Eiffel language as used in practice, obviously including Eiffel's native notation for contracts (specification elements) such as pre- and postconditions, class invariants, loop invariants and variants, and inlined assertions such as \e{check} (\b{assert} in other languages).
Object-oriented features---classes and types, multiple inheritance, polymorphism---are fully supported, and so are imperative and procedural constructs.


%----------------------------------------------------------------------------
\subsubsection{Partially supported and unsupported features}
%----------------------------------------------------------------------------

A few language features that \AutoProof does not currently fully support have a semantics that violates well-formedness conditions required for verification: 
AutoProof doesn't support specification expressions with side effects (for example, a precondition that creates an object). These are common practices in verification and recommended also in Eiffel, although programs in practice may include specification functions with side effects.
It also doesn't support the semantics of \e{once} routines (similar to \lstinline[language=Java]|static| in Java and C\#), which would require global reasoning thus breaking modularity.

Other partially supported features originate in the distinction between machine and mathematical representation of types.
Among primitive types, machine \e{INTEGER}s are fully supported (including overflows); floating-point \e{REAL}s are modeled as infinite-precision mathematical reals; strings are supported with a limited set of operations and without UTF support.
Array and list library containers with simplified interfaces are supported out of the box.
Other container types require custom specification.
Agents (function objects) are partially supported, with restrictions in their specifications.
The semantics of native \e{external} methods (no source code is available) is reduced to their specification.
We designed a translation for exceptions based on the latest draft of the Eiffel language standard (see Section~\ref{sec:m-exceptions}), but \AutoProof doesn't support it yet since the Eiffel compiler still only implements the former syntax for exceptions (and exceptions have very limited usage in Eiffel anyway).

The \AutoProof manual (Appendix~\ref{manual:lang-support}) has an overview of the level of support for all Eiffel language features.

%----------------------------------------------------------------------------
\subsubsection{Annotations for verification}
%----------------------------------------------------------------------------

Supporting effective auto-active verification requires much more than translating the input language and specification into verification conditions.
\AutoProof's verification methodology relies on annotations that are not part of the Eiffel language.
Users provide them by writing two kinds of constructs.
Annotations that are part of assertions or other specification elements use predefined dummy features with empty implementation, which \AutoProof interprets according to their semantics.
Annotations of this kind include \emph{modify} and \emph{read} clauses (specifying objects whose state may be modified or read by a routine's body).
For instance, a clause \e{modify (set)} in a routine's precondition denotes that executing the routine may modify objects in \e{set}.

Annotations that apply to whole classes or features are expressed by means of Eiffel's \e{note} clauses, which attach additional information that is ignored by the Eiffel compiler but is processed by \AutoProof.
Annotations of this kind include defining class members as \emph{ghost} (only used in specifications), procedures as \emph{lemmas} (outlining a proof using assertions and ghost-state manipulation), and which members of a class define its abstract \emph{model} (to be referred to in interface specifications).
For example \e{note status: ghost} tags as ghost the member it is attached to.
A complete list of supported annotations is available in the \AutoProof manual (Appendix~\ref{manual:annotations}).

A distinctive trait of semantic collaboration, as available to \AutoProof users, is the combination of flexible expressive annotations with useful defaults.
Expressivity offers fine-grained control over the visibility of specification elements (for example, invariant clauses can be referenced individually); defaults reduce the amount of required manual annotations in many practical cases; the combination of the two is instrumental in making \AutoProof usable on complex examples of realistic object-oriented programs.


%----------------------------------------------------------------------------
\subsubsection{Verifier's options}
%----------------------------------------------------------------------------

\AutoProof \emph{verification options} are also expressed by means of \e{note} clauses: users can disable generating boilerplate implicit contracts, skip verification of a specific class, 
disable termination checking (only verify partial correctness), and define a custom mapping of a class's type to a Boogie theory file.
See \AutoProof's manual (Appendix~\ref{manual:options}) for a complete list of features and options, and examples of usage.


%----------------------------------------------------------------------------
\subsubsection{Specification library}
%----------------------------------------------------------------------------

To support writing complex specifications, \AutoProof provides a library---called MML for Mathematical Model Library---of pre-defined abstract types.
These include mathematical structures such as sets, relations, sequences, bags (multisets), and maps.
The MML annotation style follows the model-based paradigm~\cite{POLIKARPOVA10}, which helps write abstract and concise, yet expressive, specifications.
MML's features are fully integrated in \AutoProof by means of effective mappings to Boogie background theories.
A distinctive advantage of providing mathematical types as an annotated library is that MML is \emph{extensible}: users can easily provide additional abstractions by writing annotated Eiffel classes and by linking them to background theories using custom \e{note} annotations---in the very same way existing MML classes are defined.
This is not possible in most other auto-active verifiers, where mathematical types for specification are built into the language syntax.


