%############################################################################
\section{\AutoProof in the Classroom}
\label{sec:eval-teaching}
%############################################################################

\AutoProof has been used in a Master's level software verification course at ETH Zurich in the fall semester 2014~\cite{SVCOURSE14}.
The course introduces students to different techniques and tools of software verification.
The students were provided with an introduction and a tutorial for \AutoProof, and then used \AutoProof for an exercise and a project.

%----------------------------------------------------------------------------
\subsubsection{Exercise}
%----------------------------------------------------------------------------

In the exercise, students used the web-interface of \AutoProof to solve different tasks of increasing difficulty:

\begin{enumerate}

\item The students were given a class that implements a simple counter starting from 0 and wrapping around at 59. The class was missing the precondition and several postcondition consisting of basic arithmetic properties, which students had to fill in.

\item Students were given Hoare triples which they had to encode and check in \AutoProof.

\item The third task was to add the precondition and loop invariant to an algorithm that computes the maximum of an array. The solution required a first-order loop invariant, which could be adapted from the given postcondition.

\item Given an algorithm that computes the sum and maximum element of an array, the students were asked to provide a loop invariant and postcondition about interesting properties of the algorithm.

\item In the last task, the students were given the implementation of the \emph{longest common prefix} algorithm of the FM 2012 verification competition without any contracts, together with a client of the \e{lcp} function using it for various test cases. The students were asked to write the full specification of the \e{lcp} function.

\end{enumerate}


%----------------------------------------------------------------------------
\subsubsection{Project}
%----------------------------------------------------------------------------

In small teams the students had to use both \AutoProof and Boogie~\cite{LEINO08} to implement and verify a list class. The class uses an array for the storage of elements and has a special sort routine using a different sorting algorithm based on the number and the value range of the list elements. If the list contains more than a predefined number of elements and all these elements are in a given range, the bucket-sort algorithm is used, otherwise quick-sort~\cite{CORMEN09}. The two sorting algorithms have non-trivial specifications and, depending on the encoding of the specifications, need additional lemmas to be successfully verified with \AutoProof.


%----------------------------------------------------------------------------
\subsubsection{Results}
%----------------------------------------------------------------------------

There were nine teams that finished the project.
The results of using \AutoProof are the following:
\begin{itemize}
\item Two teams specified and verified the list class completely.
\item Five teams achieved very good results and provided full specification, but failed in verifying all components of the sorting routines' postcondition.
\item Two teams performed badly, partially due to introducing inconsistencies in the specification without noticing it.
\end{itemize}

The results indicate that the students were able to successfully use \AutoProof to verify a non-trivial algorithm.
However, there were two main issues:
\begin{enumerate}
\item
Teams that achieved good results but failed to completely verify the classes had issues to help \AutoProof verify properties of \emph{sets} and \emph{sequences}  modelled with MML. Depending on the encoding used to write the specifications, a considerable amount of intermediate assertions or lemmas is necessary to verify such properties.

\item
When writing inconsistent specifications, \AutoProof will happily verify every program. There are currently no built-in facilities to check if specifications contain a contradiction.
\end{enumerate}

To remedy this situation, students need to be provided with better training on \AutoProof.
For this, we have created a new tutorial that provides a more systematic approach and more exercises for use in future courses (see Appendix~\ref{sec:ap-tutorial}).

